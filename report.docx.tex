\documentclass[a4paper,12pt]{article}
\usepackage[utf8]{inputenc}
\usepackage[english,persian]{babel}
\usepackage{geometry}
\geometry{a4paper, margin=1in}

% Adding packages for formatting and tables
\usepackage{fancyhdr}
\usepackage{lastpage}
\usepackage{graphicx}
\usepackage{enumitem}
\usepackage{amsmath}
\usepackage{setspace}
\onehalfspacing

% Header and footer setup
\pagestyle{fancy}
\fancyhf{}
\fancyhead[L]{گزارش توسعه افزونه Azure DevOps}
\fancyhead[R]{تاریخ: 10 مه 2025}
\fancyfoot[C]{\thepage\ از \pageref{LastPage}}

% Document begins
\begin{document}

\begin{titlepage}
    \centering
    \vspace*{2cm}
    {\Huge\textbf{گزارش رسمی توسعه افزونه جدید Azure DevOps برای نصب سرویس‌های ویندوزی سازگار با NServiceBus}}
    \vspace{1cm}
    {\Large تهیه‌شده توسط: سجاد صبری}
    \vspace{1cm}
    {\large تاریخ ارائه: 10 مه 2025}
    \vfill
\end{titlepage}

\section*{مقدمه}
تاکنون، تیم ما از افزونه‌ای در پلتفرم Azure DevOps برای نصب سرویس‌های ویندوزی بهره می‌برد که بر پایه دستورالعمل‌های \texttt{sc} (Service Control) عمل می‌کرد. با این حال، این روش به دلیل عدم پشتیبانی از نیازهای خاص سرویس‌های مبتنی بر NServiceBus، از جمله پیکربندی‌های پیشرفته و مدیریت فایل‌های وابسته، کارایی لازم را نداشت. به منظور رفع این محدودیت‌ها و ارتقاء فرآیندهای استقرار، پروژه توسعه یک افزونه جدید آغاز شد که با در نظر گرفتن الزامات NServiceBus طراحی و اجرا گردید.

\section*{اهداف پروژه}
\begin{itemize}
    \item جایگزینی افزونه پیشین مبتنی بر \texttt{sc} با راهکاری مدرن و سازگار با معماری NServiceBus.
    \item خودکارسازی کامل فرآیند نصب و راه‌اندازی سرویس‌های ویندوزی در چارچوب پایپلاین‌های Azure DevOps.
    \item ارائه قابلیت پشتیبانی از پیکربندی‌های اختصاصی NServiceBus، نظیر نوع پیکربندی نقطه پایانی.
    \item مدیریت بهینه فایل‌ها از آرتیفکت‌های پابلیش‌شده تا مقصد نصب.
    \item پیاده‌سازی سیستم لاگ‌گیری دقیق برای تسهیل عیب‌یابی و پایش.
\end{itemize}

\section*{اقدامات انجام‌شده}

\subsection*{تحلیل نیازها و محدودیت‌های افزونه پیشین}
افزونه قبلی با بهره‌گیری از دستورات \texttt{sc create} و \texttt{sc start}، صرفاً امکان ایجاد و راه‌اندازی سرویس‌های ساده ویندوزی را فراهم می‌ساخت. این رویکرد برای سرویس‌های NServiceBus که نیازمند اجرای فایل اجرایی با آرگومان‌های خاص (مانند \texttt{/install}) هستند، ناکافی بود. عدم توانایی مدیریت خودکار فایل‌های وابسته و کپی آن‌ها از آرتیفکت‌های بیلد به مقصد نصب، به‌عنوان یک ضعف اساسی شناسایی شد. همچنین، فقدان مکانیزم لاگ‌گیری جامع، فرآیند عیب‌یابی را با چالش مواجه می‌کرد.

\subsection*{طراحی ساختار جدید}
طراحی یک تسک جدید مبتنی بر اسکریپت PowerShell در دستور کار قرار گرفت که شامل موارد زیر بود:
\begin{itemize}
    \item کپی‌سازی تمامی فایل‌ها از مسیر پابلیش‌شده (مانند آرتیفکت‌های بیلد) به دایرکتوری نصب.
    \item شناسایی و استفاده از اولین فایل \texttt{.exe} موجود برای نصب سرویس.
    \item پشتیبانی از آرگومان‌های اختصاصی NServiceBus (مانند \texttt{/servicename} و \texttt{/endpointConfigurationType}).
\end{itemize}
فایل‌های مانیفست (\texttt{vss-extension.json} و \texttt{task.json}) برای یکپارچگی با پلتفرم Azure DevOps تدوین شدند.

\subsection*{توسعه و پیاده‌سازی فایل‌ها}
\begin{itemize}
    \item \textbf{فایل \texttt{vss-extension.json}:}
    این فایل به‌عنوان مانیفست اصلی اکستنشن عمل می‌کند و شامل شناسه (\texttt{install-windows-service-task})، نام (\texttt{Install Windows Service})، نسخه (1.1.0)، و ناشر (\texttt{SajjadSaberi}) است. تسک به‌عنوان یک مشارکت از نوع \texttt{ms.vss-distributed-task.task} ثبت شده و دایرکتوری \texttt{task} را شامل می‌شود. آیکون (\texttt{icon.png}) برای نمایش در Azure DevOps Marketplace تعریف گردید.

    \item \textbf{فایل \texttt{task.json}:}
    تسک با شناسه منحصربه‌فرد \texttt{ecb2893f-6b57-48c9-9a9e-f2b8b65d601a} و نام \texttt{InstallNServiceBusService} معرفی شده است. ورودی‌هایی نظیر \texttt{serviceName}، \texttt{displayName}، \texttt{sourcePublishedPath}، \texttt{installPath}، \texttt{serviceAccountType}، \texttt{startType}، و \texttt{endpointConfigurationType} (اختیاری) طراحی شده‌اند. اجرای تسک با استفاده از PowerShell 3 و اسکریپت \texttt{install-service.ps1} پیکربندی شده است.

    \item \textbf{فایل \texttt{install-service.ps1}:}
    این اسکریپت فرآیند نصب را به‌صورت زیر مدیریت می‌کند:
    \begin{itemize}
        \item توقف و حذف سرویس موجود (در صورت وجود) با استفاده از \texttt{Stop-Service} و \texttt{sc delete}.
        \item پاک‌سازی مسیر نصب قبلی و ایجاد دایرکتوری جدید.
        \item کپی‌سازی تمامی فایل‌ها از \texttt{sourcePublishedPath} به \texttt{installPath}.
        \item شناسایی اولین فایل \texttt{.exe} و اجرای آن با آرگومان‌های NServiceBus (مانند \texttt{/servicename} و \texttt{/displayname}).
        \item راه‌اندازی سرویس با دستور \texttt{start}.
    \end{itemize}
    لاگ‌ها در مسیر \texttt{\$env:Temp\<serviceName>-install-log.txt} ثبت می‌شوند.
\end{itemize}

\subsection*{بهینه‌سازی و اعتبارسنجی}
پارامتر \texttt{sourcePublishedPath} جایگزین \texttt{exeSourcePath} شد تا از آرتیفکت‌های پابلیش‌شده پشتیبانی کند. ترتیب ورودی‌ها در \texttt{task.json} تنظیم شد تا \texttt{sourcePublishedPath} پیش از \texttt{installPath} قرار گیرد (به‌منظور بهبود تجربه کاربری). نسخه از \texttt{1.0.0} به \texttt{1.1.0} ارتقا یافت تا بهبودهای اعمال‌شده (مانند کپی همه فایل‌ها) منعکس شود. اسکریپت با افزودن بررسی وجود مسیر و فایل \texttt{.exe} بهینه شد تا از وقوع خطاها جلوگیری به عمل آید.

\subsection*{تهیه مستندات}
فایل \texttt{README.md} به دو زبان فارسی و انگلیسی تدوین شد:
\begin{itemize}
    \item شامل بخش‌هایی نظیر مرور کلی، پیش‌نیازها (مانند دسترسی به آرتیفکت‌ها و محیط ویندوزی)، نحوه استفاده، و نکات کلیدی (مانند ضرورت وجود فایل \texttt{.exe} و محل لاگ‌ها).
    \item این مستندات به‌عنوان راهنمایی جامع برای کاربران طراحی شده‌اند.
\end{itemize}

\section*{مقایسه با افزونه پیشین}
\begin{itemize}
    \item \textbf{افزونه پیشین (بر پایه \texttt{sc}):}
    صرفاً از دستورات \texttt{sc} برای ایجاد و راه‌اندازی سرویس استفاده می‌کرد. فاقد قابلیت مدیریت خودکار فایل‌ها و نیازمند کپی دستی بود. پشتیبانی از آرگومان‌های NServiceBus را ارائه نمی‌داد. فاقد سیستم لاگ‌گیری بود.

    \item \textbf{افزونه جدید:}
    از اسکریپت PowerShell برای مدیریت جامع فایل‌ها و نصب سرویس بهره می‌برد. تمامی فایل‌ها از آرتیفکت‌های پابلیش‌شده کپی می‌شوند. از آرگومان‌های اختصاصی NServiceBus پشتیبانی می‌کند. سیستم لاگ‌گیری دقیق برای پایش و عیب‌یابی ارائه می‌دهد.
\end{itemize}

\section*{نتیجه‌گیری}
افزونه جدید با رفع کاستی‌های نسخه پیشین، راهکاری جامع و انعطاف‌پذیر برای نصب سرویس‌های NServiceBus در محیط Azure DevOps فراهم آورده است. ارتقا به نسخه 1.1.0، بهبودهایی نظیر کپی‌سازی خودکار فایل‌ها و مستندات دو زبانه را نشان می‌دهد. این پروژه با موفقیت الزامات تیم را برآورده کرده و برای انتشار یا استفاده در محیط‌های عملیاتی آماده است.

\section*{پیشنهادات برای توسعه آتی}
\begin{itemize}
    \item افزودن امکان انتخاب فایل \texttt{.exe} خاص به‌جای استفاده از اولین فایل.
    \item پشتیبانی از نصب چندین سرویس در یک تسک واحد.
    \item به‌کارگیری متغیرهای مخفی (Secrets) برای افزایش امنیت مدیریت رمزهای عبور.
\end{itemize}

\end{document}